\documentclass[11pt, a4paper, leqno]{article}
\usepackage{a4wide}
\usepackage[T1]{fontenc}
\usepackage[utf8]{inputenc}
\usepackage{float, afterpage, rotating, graphicx}
\usepackage{epstopdf}
\usepackage{longtable, booktabs, tabularx}
\usepackage{fancyvrb, moreverb, relsize}
\usepackage{eurosym, calc}
% \usepackage{chngcntr}
\usepackage{amsmath, amssymb, amsfonts, amsthm, bm}
\usepackage{caption}
\usepackage{mdwlist}
\usepackage{xfrac}
\usepackage{setspace}
\usepackage{xcolor}
\usepackage{subcaption}
\usepackage{minibox}
\usepackage{adjustbox}
\usepackage{comment}
% \usepackage{pdf14} % Enable for Manuscriptcentral -- can't handle pdf 1.5
% \usepackage{endfloat} % Enable to move tables / figures to the end. Useful for some submissions.


\usepackage[
    natbib=true,
    bibencoding=inputenc,
    bibstyle=authoryear-ibid,
    citestyle=authoryear-comp,
    maxcitenames=3,
    maxbibnames=10,
    useprefix=false,
    sortcites=true,
    backend=biber
]{biblatex}
\AtBeginDocument{\toggletrue{blx@useprefix}}
\AtBeginBibliography{\togglefalse{blx@useprefix}}
\setlength{\bibitemsep}{1.5ex}
\addbibresource{refs.bib}





\usepackage[unicode=true]{hyperref}
\hypersetup{
    colorlinks=true,
    linkcolor=black,
    anchorcolor=black,
    citecolor=black,
    filecolor=black,
    menucolor=black,
    runcolor=black,
    urlcolor=black
}


\widowpenalty=10000
\clubpenalty=10000

\setlength{\parskip}{1ex}
\setlength{\parindent}{0ex}
\setstretch{1.5}


\begin{document}

\title{covid-19 Mobility\thanks{Bahar Coskun, Timo Haller, Anto Marcinkovic, Universitaet Bonn. Email: \href{mailto:s6bacosk@uni-bonn.de, s6tohall@uni-bonn.de, s3anmarc@uni-bonn.de}{\nolinkurl{s6bacosk [at] uni-bonn [dot] de, s6tohall [at] uni-bonn [dot] de, s3anmarc [at] uni-bonn [dot] de}}.}}

\author{Bahar Coskun, Timo Haller, Anto Marcinkovic}

\date{
    {\bf Preliminary -- please do not quote}
    \\[1ex]
    \today
}

\maketitle


\begin{abstract}
    Some abstract here.
\end{abstract}
\clearpage

\tableofcontents
\clearpage

\section{Introduction} % (fold)
\label{sec:introduction}

If you are using this template, please cite this item from the references: \citet{GaudeckerEconProjectTemplates}

\newpage

\section{Regression}

This short paragraph provides an explanation on our regression analysis.
In our analysis we want to identify the effect of lockdown fatigue, which means people reduce their mobility less after a long lockdown history.
Therefore we estimate the following regression model:


$\textit{mobility}_{t} = \beta_{0} + \beta_{1} \textit{LD1}_{t} + \beta_{1} \textit{LD1}_{t} + \beta_{1} \textit{LD1}_{t} \times + \textit{LD1\_Duration}_{t} + \beta_{2} \textit{LD2}_{t} \times \textit{LD2\_Duration}_{t} + \beta_{5} + \gamma \mathbf{X} + \epsilon_{i}$

whereby $\textit{LD1}_{t} $ and $\textit{LD2}_{t}$ denote the second and first lockdown. The variable $\textit{LD1\_Duration}_{t}$ is the duration of the first lockdown (equivalently for $\textit{LD1\_Duration}_{t}$). Furthermore, $\textit{stringency}$ controls for the stringency index. Additional control variables, in particular the infection case number, are summarized in the $\mathbf{X}$ vector. \\
The marginal effects of interest are $\beta_{2}$ and $\beta_{3}$. We expect $\beta_{3}$ to be positive and $\beta_{2}$ to be zero or even negative. The following section shows the tables for different mobility measures as dependent variables.





\newpage

\begin{table}
	\centering
	\caption{Lockdown Fatigue for Workplace Mobility (7 Days Moving Average)}
	\scalebox{0.8}{%
	\begin{tabular}{@{\extracolsep{5pt}}lccccc}
\\[-1.8ex]\hline
\hline \\[-1.8ex]
& \multicolumn{5}{c}{\textit{}} \
\cr \cline{5-6}
\\[-1.8ex] & (1) & (2) & (3) & (4) & (5) \\
\hline \\[-1.8ex]
 1st Lockdown Duration x Stringency & 0.017$^{}$ & 0.069$^{***}$ & 0.069$^{***}$ & 0.069$^{***}$ & 0.071$^{***}$ \\
  & (0.012) & (0.012) & (0.011) & (0.011) & (0.011) \\
 2nd Lockdown Duration x Stringency & 0.085$^{}$ & 0.359$^{***}$ & 0.359$^{***}$ & 0.325$^{***}$ & 0.338$^{***}$ \\
  & (0.085) & (0.090) & (0.082) & (0.085) & (0.088) \\
 Stringency & -0.512$^{***}$ & -0.382$^{***}$ & -0.475$^{***}$ & -0.492$^{***}$ & -0.486$^{***}$ \\
  & (0.043) & (0.040) & (0.038) & (0.040) & (0.040) \\
 1st Lockdown & -12.398$^{***}$ & 43.355$^{***}$ & 40.223$^{***}$ & 40.972$^{***}$ & 41.660$^{***}$ \\
  & (4.011) & (7.472) & (6.837) & (6.857) & (6.855) \\
 2nd Lockdown & -13.813$^{***}$ & 187.163$^{***}$ & 184.031$^{***}$ & 170.479$^{***}$ & 142.364$^{***}$ \\
  & (2.753) & (34.878) & (31.871) & (32.989) & (34.075) \\
 1st Lockdown Duration & -1.349$^{}$ & -4.816$^{***}$ & -4.816$^{***}$ & -4.769$^{***}$ & -4.912$^{***}$ \\
  & (0.981) & (0.964) & (0.881) & (0.880) & (0.871) \\
 2nd Lockdown Duration & -6.899$^{}$ & -29.617$^{***}$ & -29.617$^{***}$ & -26.616$^{***}$ & -27.766$^{***}$ \\
  & (7.124) & (7.500) & (6.853) & (7.113) & (7.375) \\
 1st Lockdown x Stringency & & -1.030$^{***}$ & -0.937$^{***}$ & -0.974$^{***}$ & -0.976$^{***}$ \\
  & & (0.122) & (0.112) & (0.115) & (0.117) \\
 2nd Lockdown x Stringency & & -2.522$^{***}$ & -2.428$^{***}$ & -2.369$^{***}$ & -1.979$^{***}$ \\
  & & (0.431) & (0.394) & (0.395) & (0.410) \\
 Light Lockdown & & & 9.899$^{***}$ & 5.017$^{}$ & 9.558$^{***}$ \\
  & & & (1.159) & (3.246) & (3.515) \\
 Light Lockdown Duration & & & & -0.026$^{}$ & -0.009$^{}$ \\
  & & & & (0.074) & (0.087) \\
 New cases & & & & 0.000$^{}$ & 0.001$^{}$ \\
  & & & & (0.000) & (0.001) \\
 New Cases Squared & & & & & -0.000$^{*}$ \\
  & & & & & (0.000) \\
 New Cases Cubic & & & & & 0.000$^{**}$ \\
  & & & & & (0.000) \\
 Intercept & 10.065$^{***}$ & 2.653$^{}$ & 5.785$^{***}$ & 6.245$^{***}$ & 5.703$^{***}$ \\
  & (2.488) & (2.335) & (2.165) & (2.184) & (2.164) \\
\hline \\[-1.8ex]
 Observations & 374 & 374 & 374 & 374 & 374 \\
 $R^2$ & 0.609 & 0.695 & 0.746 & 0.748 & 0.755 \\
 Adjusted $R^2$ & 0.601 & 0.688 & 0.739 & 0.740 & 0.746 \\
 Residual Std. Error & 8.985 & 7.954 & 7.267 & 7.260 & 7.170  \\
 F Statistic & 81.372$^{***}$  & 92.218$^{***}$  & 106.705$^{***}$  & 89.334$^{***}$  & 85.317$^{***}$  \\
\hline
\hline \\[-1.8ex]
\textit{Note:} & \multicolumn{5}{r}{$^{*}$p$<$0.1; $^{**}$p$<$0.05; $^{***}$p$<$0.01} \\
\end{tabular}

}

\end{table}


\newpage

\begin{table}
	\centering
	\caption{Lockdown Fatigue for Transit Station Mobility (7 Days Moving Average)}
	\scalebox{0.8}{%
		\begin{tabular}{@{\extracolsep{5pt}}lccccc}
\\[-1.8ex]\hline
\hline \\[-1.8ex]
& \multicolumn{5}{c}{\textit{}} \
\cr \cline{5-6}
\\[-1.8ex] & (1) & (2) & (3) & (4) & (5) \\
\hline \\[-1.8ex]
 1st Lockdown Duration x Stringency & 0.043$^{***}$ & 0.093$^{***}$ & 0.093$^{***}$ & 0.093$^{***}$ & 0.092$^{***}$ \\
  & (0.009) & (0.010) & (0.009) & (0.008) & (0.008) \\
 2nd Lockdown Duration x Stringency & -0.047$^{}$ & 0.009$^{}$ & 0.009$^{}$ & 0.109$^{*}$ & 0.037$^{}$ \\
  & (0.067) & (0.070) & (0.065) & (0.064) & (0.063) \\
 Stringency & -0.752$^{***}$ & -0.642$^{***}$ & -0.576$^{***}$ & -0.526$^{***}$ & -0.492$^{***}$ \\
  & (0.033) & (0.031) & (0.030) & (0.030) & (0.028) \\
 1st Lockdown & -15.392$^{***}$ & 37.741$^{***}$ & 39.971$^{***}$ & 37.835$^{***}$ & 35.114$^{***}$ \\
  & (3.148) & (5.823) & (5.416) & (5.170) & (4.896) \\
 2nd Lockdown & -14.716$^{***}$ & 30.824$^{}$ & 33.054$^{}$ & 72.058$^{***}$ & 22.587$^{}$ \\
  & (2.160) & (27.182) & (25.248) & (24.874) & (24.337) \\
 1st Lockdown Duration & -3.258$^{***}$ & -6.618$^{***}$ & -6.618$^{***}$ & -6.755$^{***}$ & -6.808$^{***}$ \\
  & (0.770) & (0.751) & (0.698) & (0.664) & (0.622) \\
 2nd Lockdown Duration & 3.971$^{}$ & -0.637$^{}$ & -0.637$^{}$ & -9.271$^{*}$ & -3.089$^{}$ \\
  & (5.591) & (5.845) & (5.429) & (5.363) & (5.268) \\
 1st Lockdown x Stringency & & -0.982$^{***}$ & -1.049$^{***}$ & -0.941$^{***}$ & -0.834$^{***}$ \\
  & & (0.095) & (0.088) & (0.086) & (0.084) \\
 2nd Lockdown x Stringency & & -0.595$^{*}$ & -0.662$^{**}$ & -0.833$^{***}$ & -0.330$^{}$ \\
  & & (0.336) & (0.312) & (0.298) & (0.293) \\
 Light Lockdown & & & -7.049$^{***}$ & 7.133$^{***}$ & 11.931$^{***}$ \\
  & & & (0.918) & (2.448) & (2.511) \\
 Light Lockdown Duration & & & & 0.070$^{}$ & -0.071$^{}$ \\
  & & & & (0.056) & (0.062) \\
 New cases & & & & -0.001$^{***}$ & -0.002$^{***}$ \\
  & & & & (0.000) & (0.000) \\
 New Cases Squared & & & & & 0.000$^{}$ \\
  & & & & & (0.000) \\
 New Cases Cubic & & & & & 0.000$^{}$ \\
  & & & & & (0.000) \\
 Intercept & 21.722$^{***}$ & 15.446$^{***}$ & 13.216$^{***}$ & 11.873$^{***}$ & 11.357$^{***}$ \\
  & (1.953) & (1.820) & (1.715) & (1.647) & (1.546) \\
\hline \\[-1.8ex]
 Observations & 374 & 374 & 374 & 374 & 374 \\
 $R^2$ & 0.836 & 0.874 & 0.892 & 0.903 & 0.915 \\
 Adjusted $R^2$ & 0.833 & 0.871 & 0.889 & 0.900 & 0.912 \\
 Residual Std. Error & 7.051 & 6.199 & 5.757 & 5.474 & 5.121  \\
 F Statistic & 267.388$^{***}$  & 281.293$^{***}$  & 299.381$^{***}$  & 279.318$^{***}$  & 298.644$^{***}$  \\
\hline
\hline \\[-1.8ex]
\textit{Note:} & \multicolumn{5}{r}{$^{*}$p$<$0.1; $^{**}$p$<$0.05; $^{***}$p$<$0.01} \\
\end{tabular}

	}
\end{table}

\newpage

\begin{table}
	\centering
	\caption{Lockdown Fatigue for Retail and Recreation Mobility (7 Days Moving Average)}
	\scalebox{0.8}{%
		\begin{tabular}{@{\extracolsep{5pt}}lccccc}
\\[-1.8ex]\hline
\hline \\[-1.8ex]
& \multicolumn{5}{c}{\textit{}} \
\cr \cline{5-6}
\\[-1.8ex] & (1) & (2) & (3) & (4) & (5) \\
\hline \\[-1.8ex]
 1st Lockdown Duration x Stringency & -0.002$^{**}$ & -0.002$^{**}$ & -0.002$^{**}$ & -0.003$^{***}$ & -0.003$^{***}$ \\
  & (0.001) & (0.001) & (0.001) & (0.001) & (0.001) \\
 2nd Lockdown Duration x Stringency & -0.010$^{***}$ & 0.009$^{*}$ & 0.009$^{*}$ & 0.005$^{}$ & 0.010$^{***}$ \\
  & (0.002) & (0.005) & (0.005) & (0.004) & (0.003) \\
 Stringency & 0.089$^{***}$ & 0.102$^{***}$ & 0.081$^{***}$ & -0.229$^{***}$ & -0.444$^{***}$ \\
  & (0.029) & (0.028) & (0.031) & (0.035) & (0.033) \\
 1st Lockdown & -22.727$^{***}$ & -0.004$^{***}$ & -0.004$^{***}$ & -0.002$^{***}$ & 0.014$^{}$ \\
  & (3.131) & (0.000) & (0.000) & (0.000) & (0.010) \\
 2nd Lockdown & -24.241$^{***}$ & 26.192$^{**}$ & 26.844$^{**}$ & 28.117$^{***}$ & -3.946$^{}$ \\
  & (3.191) & (11.511) & (11.810) & (9.762) & (8.823) \\
 1st Lockdown Duration & -0.000$^{**}$ & -0.000$^{**}$ & -0.000$^{**}$ & -0.000$^{***}$ & -0.000$^{***}$ \\
  & (0.000) & (0.000) & (0.000) & (0.000) & (0.000) \\
 2nd Lockdown Duration & -0.078$^{}$ & -0.809$^{***}$ & -0.836$^{***}$ & -2.211$^{***}$ & -0.962$^{***}$ \\
  & (0.066) & (0.173) & (0.196) & (0.192) & (0.178) \\
 1st Lockdown x Stringency & & -0.288$^{***}$ & -0.286$^{***}$ & -0.127$^{***}$ & 0.010$^{}$ \\
  & & (0.038) & (0.038) & (0.034) & (0.030) \\
 2nd Lockdown x Stringency & & -1.310$^{***}$ & -1.289$^{***}$ & -0.914$^{***}$ & -0.482$^{**}$ \\
  & & (0.288) & (0.288) & (0.240) & (0.203) \\
 Light Lockdown & & & -3.654$^{}$ & 20.763$^{***}$ & 25.926$^{***}$ \\
  & & & (3.191) & (3.214) & (2.680) \\
 Light Lockdown Duration & & & 0.027$^{}$ & 0.628$^{***}$ & 0.170$^{**}$ \\
  & & & (0.093) & (0.089) & (0.080) \\
 New cases & & & & -0.003$^{***}$ & -0.008$^{***}$ \\
  & & & & (0.000) & (0.001) \\
 New Cases Squared & & & & & 0.000$^{***}$ \\
  & & & & & (0.000) \\
 New Cases Cubic & & & & & -0.000$^{}$ \\
  & & & & & (0.000) \\
 Intercept & -3.722$^{***}$ & -4.262$^{***}$ & -2.765$^{*}$ & 18.682$^{***}$ & 35.343$^{***}$ \\
  & (1.421) & (1.390) & (1.676) & (2.127) & (2.224) \\
\hline \\[-1.8ex]
 Observations & 389 & 389 & 389 & 389 & 389 \\
 $R^2$ & 0.676 & 0.693 & 0.695 & 0.792 & 0.865 \\
 Adjusted $R^2$ & 0.671 & 0.687 & 0.688 & 0.787 & 0.861 \\
 Residual Std. Error & 11.189 & 10.911 & 10.903 & 9.012 & 7.272  \\
 F Statistic & 133.106$^{***}$  & 122.936$^{***}$  & 96.053$^{***}$  & 144.194$^{***}$  & 201.598$^{***}$  \\
\hline
\hline \\[-1.8ex]
\textit{Note:} & \multicolumn{5}{r}{$^{*}$p$<$0.1; $^{**}$p$<$0.05; $^{***}$p$<$0.01} \\
\end{tabular}

	}
\end{table}

\newpage

\begin{table}
	\centering
	\caption{Lockdown Fatigue for Grocery and Pharmacy Mobility (7 Days Moving Average)}
	\scalebox{0.8}{%
		\begin{tabular}{@{\extracolsep{5pt}}lccccc}
\\[-1.8ex]\hline
\hline \\[-1.8ex]
& \multicolumn{5}{c}{\textit{}} \
\cr \cline{5-6}
\\[-1.8ex] & (1) & (2) & (3) & (4) & (5) \\
\hline \\[-1.8ex]
 1st Lockdown Duration x Stringency & 0.001$^{***}$ & 0.001$^{***}$ & 0.001$^{***}$ & 0.001$^{***}$ & 0.001$^{***}$ \\
  & (0.000) & (0.000) & (0.000) & (0.000) & (0.000) \\
 2nd Lockdown Duration x Stringency & -0.007$^{***}$ & 0.004$^{**}$ & 0.004$^{**}$ & 0.004$^{**}$ & 0.004$^{***}$ \\
  & (0.001) & (0.002) & (0.002) & (0.002) & (0.002) \\
 Stringency & 0.056$^{***}$ & 0.063$^{***}$ & 0.054$^{***}$ & 0.008$^{}$ & -0.050$^{***}$ \\
  & (0.011) & (0.010) & (0.011) & (0.015) & (0.015) \\
 1st Lockdown & -12.954$^{***}$ & -0.002$^{***}$ & -0.002$^{***}$ & -0.002$^{***}$ & -0.004$^{}$ \\
  & (1.185) & (0.000) & (0.000) & (0.000) & (0.005) \\
 2nd Lockdown & -7.542$^{***}$ & 21.254$^{***}$ & 19.714$^{***}$ & 19.905$^{***}$ & 3.038$^{}$ \\
  & (1.208) & (4.204) & (4.281) & (4.169) & (4.049) \\
 1st Lockdown Duration & 0.000$^{***}$ & 0.000$^{***}$ & 0.000$^{***}$ & 0.000$^{***}$ & 0.000$^{***}$ \\
  & (0.000) & (0.000) & (0.000) & (0.000) & (0.000) \\
 2nd Lockdown Duration & 0.068$^{***}$ & -0.349$^{***}$ & -0.426$^{***}$ & -0.632$^{***}$ & -0.189$^{**}$ \\
  & (0.025) & (0.063) & (0.071) & (0.082) & (0.082) \\
 1st Lockdown x Stringency & & -0.164$^{***}$ & -0.163$^{***}$ & -0.139$^{***}$ & -0.106$^{***}$ \\
  & & (0.014) & (0.014) & (0.014) & (0.014) \\
 2nd Lockdown x Stringency & & -0.748$^{***}$ & -0.739$^{***}$ & -0.683$^{***}$ & -0.435$^{***}$ \\
  & & (0.105) & (0.104) & (0.102) & (0.093) \\
 Light Lockdown & & & -3.351$^{***}$ & 0.299$^{}$ & 3.097$^{**}$ \\
  & & & (1.157) & (1.372) & (1.230) \\
 Light Lockdown Duration & & & 0.077$^{**}$ & 0.167$^{***}$ & 0.037$^{}$ \\
  & & & (0.034) & (0.038) & (0.037) \\
 New cases & & & & -0.000$^{***}$ & -0.001$^{***}$ \\
  & & & & (0.000) & (0.000) \\
 New Cases Squared & & & & & -0.000$^{}$ \\
  & & & & & (0.000) \\
 New Cases Cubic & & & & & 0.000$^{***}$ \\
  & & & & & (0.000) \\
 Intercept & -3.748$^{***}$ & -4.057$^{***}$ & -3.413$^{***}$ & -0.206$^{}$ & 3.932$^{***}$ \\
  & (0.538) & (0.508) & (0.608) & (0.908) & (1.021) \\
\hline \\[-1.8ex]
 Observations & 389 & 389 & 389 & 389 & 389 \\
 $R^2$ & 0.620 & 0.665 & 0.672 & 0.690 & 0.768 \\
 Adjusted $R^2$ & 0.615 & 0.659 & 0.664 & 0.682 & 0.761 \\
 Residual Std. Error & 4.236 & 3.985 & 3.952 & 3.848 & 3.338  \\
 F Statistic & 104.084$^{***}$  & 108.033$^{***}$  & 86.375$^{***}$  & 84.141$^{***}$  & 103.775$^{***}$  \\
\hline
\hline \\[-1.8ex]
\textit{Note:} & \multicolumn{5}{r}{$^{*}$p$<$0.1; $^{**}$p$<$0.05; $^{***}$p$<$0.01} \\
\end{tabular}

	}
\end{table}

\newpage

\begin{table}
	\centering
	\caption{Lockdown Fatigue for Residential Mobility (7 Days Moving Average)}
	\scalebox{0.8}{%
		\begin{tabular}{@{\extracolsep{5pt}}lccccc}
\\[-1.8ex]\hline
\hline \\[-1.8ex]
& \multicolumn{5}{c}{\textit{}} \
\cr \cline{5-6}
\\[-1.8ex] & (1) & (2) & (3) & (4) & (5) \\
\hline \\[-1.8ex]
 1st Lockdown Duration x Stringency & -0.010$^{***}$ & -0.027$^{***}$ & -0.027$^{***}$ & -0.027$^{***}$ & -0.028$^{***}$ \\
  & (0.003) & (0.003) & (0.003) & (0.003) & (0.003) \\
 2nd Lockdown Duration x Stringency & 0.022$^{}$ & -0.011$^{}$ & -0.011$^{}$ & -0.026$^{}$ & -0.033$^{}$ \\
  & (0.022) & (0.023) & (0.023) & (0.024) & (0.024) \\
 Stringency & 0.175$^{***}$ & 0.136$^{***}$ & 0.130$^{***}$ & 0.122$^{***}$ & 0.121$^{***}$ \\
  & (0.011) & (0.010) & (0.011) & (0.011) & (0.011) \\
 1st Lockdown & 6.716$^{***}$ & -11.375$^{***}$ & -11.582$^{***}$ & -11.276$^{***}$ & -11.607$^{***}$ \\
  & (1.057) & (1.935) & (1.930) & (1.924) & (1.896) \\
 2nd Lockdown & 5.504$^{***}$ & -19.996$^{**}$ & -20.203$^{**}$ & -26.048$^{***}$ & -15.092$^{}$ \\
  & (0.725) & (9.032) & (8.996) & (9.257) & (9.427) \\
 1st Lockdown Duration & 0.795$^{***}$ & 1.936$^{***}$ & 1.936$^{***}$ & 1.956$^{***}$ & 2.015$^{***}$ \\
  & (0.258) & (0.250) & (0.249) & (0.247) & (0.241) \\
 2nd Lockdown Duration & -1.861$^{}$ & 0.875$^{}$ & 0.875$^{}$ & 2.165$^{}$ & 2.747$^{}$ \\
  & (1.877) & (1.942) & (1.934) & (1.996) & (2.040) \\
 1st Lockdown x Stringency & & 0.334$^{***}$ & 0.341$^{***}$ & 0.325$^{***}$ & 0.327$^{***}$ \\
  & & (0.031) & (0.032) & (0.032) & (0.033) \\
 2nd Lockdown x Stringency & & 0.326$^{***}$ & 0.333$^{***}$ & 0.359$^{***}$ & 0.204$^{*}$ \\
  & & (0.112) & (0.111) & (0.111) & (0.113) \\
 Light Lockdown & & & 0.654$^{**}$ & -1.572$^{*}$ & -3.385$^{***}$ \\
  & & & (0.327) & (0.911) & (0.973) \\
 Light Lockdown Duration & & & & -0.007$^{}$ & -0.016$^{}$ \\
  & & & & (0.021) & (0.024) \\
 New cases & & & & 0.000$^{**}$ & -0.000$^{}$ \\
  & & & & (0.000) & (0.000) \\
 New Cases Squared & & & & & 0.000$^{***}$ \\
  & & & & & (0.000) \\
 New Cases Cubic & & & & & -0.000$^{***}$ \\
  & & & & & (0.000) \\
 Intercept & -6.267$^{***}$ & -4.076$^{***}$ & -3.869$^{***}$ & -3.655$^{***}$ & -3.438$^{***}$ \\
  & (0.656) & (0.605) & (0.611) & (0.613) & (0.599) \\
\hline \\[-1.8ex]
 Observations & 374 & 374 & 374 & 374 & 374 \\
 $R^2$ & 0.796 & 0.846 & 0.848 & 0.851 & 0.859 \\
 Adjusted $R^2$ & 0.792 & 0.843 & 0.844 & 0.846 & 0.854 \\
 Residual Std. Error & 2.368 & 2.060 & 2.051 & 2.037 & 1.984  \\
 F Statistic & 204.046$^{***}$  & 222.973$^{***}$  & 202.732$^{***}$  & 171.851$^{***}$  & 168.943$^{***}$  \\
\hline
\hline \\[-1.8ex]
\textit{Note:} & \multicolumn{5}{r}{$^{*}$p$<$0.1; $^{**}$p$<$0.05; $^{***}$p$<$0.01} \\
\end{tabular}

	}
\end{table}

\clearpage

\setstretch{1}
\printbibliography
\setstretch{1.5}




% \appendix

% The chngctr package is needed for the following lines.
% \counterwithin{table}{section}
% \counterwithin{figure}{section}

\end{document}
